Gravitational microlensing of quasars, reviewed in
Section~\ref{sec:microlensing} below, is a phenomenon seen in quasars
that have been multiply imaged by galaxies.  Individual images in a
multiple-image system can undergo sharp and seemingly random
brightness changes.  This can occur, even for constant intrinic
luminosity, as a consequence of two things:....

-------------------------------------------------------------

In Section~\ref{sec:numerics} we carry out source fitting to
lightcurves, with both noise and systematic errors are present.  We
generate a mock lightcurves by taking a crescent source across the
path AB in Figure~\ref{fig:magnification_map}, and then adding noise.
The path simulates crossing a clean but not ideal fold.  We then fit
this lightcurve to templates from both crescent and circular-Gaussian
sources across the path CB.  That is, the templates come from a
similar but not identical caustic, thus deliberately generating a
systematic error.  Despite this systematic error, a crescent source
fits the data, while a Gaussian source is rejected from the $\chi^2$
value.

--------------------------------------------

In the case of a high-quality lightcurve, the parameters of a crescent
source can be estimated by simply identifying signatures on the
lightcurve corresponding to transits of the particular source features
across the caustic.  This simple idea can be supplemented with
statistics.  We do so in Section~\ref{sec:numerics}, where we take a
realistic magnification map generated by the microlensing code
from \cite{1999A&A...346L...5W}, compute a caustic-crossing lightcurve
from it, and then fit the lightcurve with other caustic-crossing
templates.  The conclusion is that a crescent source can be
distinguised from a circular source, on the basis of caustic-crossing
lightcurves, if the caustic is a clean but not ideal fold.

-----------------------------------------------------------------
