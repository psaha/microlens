\section{introduction}
The neighbourhood of the galactic centres are mysterious places. There is very little information available to accurately map the geometric structure close to the horizon of the supermassive black holes at the galactic centre. Observing these places with great spatial resolution is one obvious solution, however, requires high quality data. The future generation telescope, like mm-VLBI, EHT etc, are capable of resolving these structures. Nevertheless, there is an alternative solution for this problem, indirect mapping of the geometry. 

Microlensing provides us with an opportunity to detect the signals, comprehensively detectable, and model the signal as the function of the source shape of the source, its internal brightness and the lensing mass distribution. Now, if the source is a quasar, the radiations are mainly dominated by the accretion disc close to the supermassive black hole horizon.

The motivation of this work is to demonstrate if one assumes a parametric form for the shape (surface brightness 2D) of the accretion disk of the supermassive black hole, is it possible to recover the 2D brightness profile by looking at the brightness of the source over a range of time.
