\section{Microlensing code and magnification maps}
For the numerical computations of lightcurves for the models a and b, the microlensing code by Joachim Wambsgans was used, which is described in Wambsganss 1999. This code is based on the hierachical tree technique to calculate the gravitational effect of the lensing masses. The underlying idea is that a microlensing situation can be separated into three planes, the observer, the lens plane and the source plane (assuming the thin lens approximation to be valid). The actual effect, namely the distortion of lightrays emitted from a distant source, on the way to the observer, is numerically computed by treating this situation from the opposite side: lightrays are shot backward from the observer to lens plane, where they become deflected according to the angle, which is given by the 2-dim. mass distribution in the lens plane.
\begin{equation}
\vec{\alpha}_{i}=\frac{4G}{c^{2}} \frac{D_{LS}}{D_{S}}\sum_{j=1}^{n}M_j \frac{\vec{r}_{ij}}{r^2_{ij}} 
\end{equation}  
For the computation of the individual deflection angles the tree idea is used. This means that the positions of all lensing masses are sorted into a 2-dim. grid (in the lens plane). This grid is subdivided into 4 smaller squares until every cell contains only one mass. For the actual calculations of the deflection angle not every mass' gravitational influence is treated equally. Instead it is made use of the fact that the influence of lenses on the considered lightray falls off with $r^{-1}$ from the point where the ray hits the lens plane. Hence masses situated at higher distances, can be grouped together in the calculation of their gravitational potential, by multipole moments. This way the amount of computation time is reduced. \\
When the deflection angles are determined, the deflected lightrays are traced to the source plane and a magnification map is created by counting the number of rays which hit each of its pixels. \\\\
Once the map is created, the lightcurve can be obtained by specifying the transition path of the source within the map. The code thereby convolves the function describing the shape of the source with the magnification pattern of the map for each time step (see equation (6)). Note that in the most observational cases in microlensing, both lens and source are moving and further that the lens configuration, and with it the magnification pattern, is changing with time. While the first subtlety i taken care of by a coordinate transformation in this analysis, for the second one the lens configuration is assumed to be constant in time.  \\\\
For the model a it was desirable to work with a caustic geometry as simple as possible. Therefore only two equal point like masses were used as lenses, to produce the magnification map depicted in figure (...). The magnification maps calculated by the microlensing code are in units of Einstein radii, hence they are principally scale free. Physical scale is introduced to them by multiplying the pixels by the correct distance factor. The Einstein angle is given by   
\begin{equation}
\theta_{E}=\sqrt{\frac{4GM}{c^{2}} \frac{D_{LS}}{D_{S}D_{L}}} 
\end{equation}  
and therefore depends on the four parameters $M,D_{S},D_{L},D_{LS}$. For the cosmological application or quasar lensing $D_{L}$ is of order 100 Mps to Gpc and $D_{S}$ of order Gpc.\\   
The scale of the magnification map can be converted into a physical length by multiplying it by a factor of $ \sqrt{\frac{D_{LS}}{D_{S}D_{L}}}$.\\\\
For the next step, the computation of the actual lightcurve, the code was modified to also allow for crescent shaped images specified through the parameter set $R_p,R_n,a,b$. In the original version of the code gaussian and disk shaped images where already implemented. Those are completely characterized by one parameter $R_p$. The values of this parameters are specified in pixel units corresponding to the magnification map. Further one needs to specify the start and end point coordinates of the path which the center of the source image follows in the magnification map (see the depiction in figure (...)).  ---Number of timesteps, convolution----
