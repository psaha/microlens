
\section{Theory}

\subsection{Basic microlensing theory}

We start by introducing in a succint manner the gravitational lensing theory that is relevant for microlensing in general and for the scope of the
present paper in particular. For a detailed presentation of the theory the reader is referrend to the books and articles that represent the references
of the current section (book by Schneider and others).\\

The general microlensing equation can be written as:
\begin{equation}
\vec{y} =   
 \begin{pmatrix}
  1-\gamma - k & 0 \\
  0 & 1 + \gamma -k \\
 \end{pmatrix} 
\vec{x} - \sum_{i=1}^{n} m_i \frac{\vec{x} - \vec{x_i}}{\left( \vec{x} - \vec{x_i} \right)^2}.
\end{equation} 

The equation is valid for a set of $\it{n}$ point (Schwarzschild) lenses placed at coordinates $\vec{x_i}$. $\vec{y}$ represents the coordinates in the source plane for the respective ray, while 
$\vec{x}$ represents the coordinates in the lens plane. Furthermore, $\gamma$ and $k$ are the shear and surface mass density.\\

For the lens mapping a Jacobian can be defined:
\begin{equation} 
A(\vec{x}) = 
 \begin{pmatrix}
  \frac{\partial{y_1}}{\partial{x_1}} & \frac{\partial{y_1}}{\partial{x_2}} \\
  \frac{\partial{y_2}}{\partial{x_1}} & \frac{\partial{y_2}}{\partial{x_2}} \\
 \end{pmatrix}. 
\end{equation}
The determinant of $A(\vec{x})$ is related to the local magnification factor through the equation:

\begin{equation}
\mu(\vec{x}) = \left| \frac{1}{det(A(\vec{x}))} \right|
\end{equation}

According to this equation, from an infinitesimally small source positioned at $\vec{x}$ in the lens plane an observed will receive a flux $\mu(\vec{x}) dF$ instead of the flux $dF$ that would be
observed in the absence of the gravitational lens. Therefore the source will dim or brighten depending on whether $\mu(\vec{x})$ is subunitary or supraunitary.\\

In general, for a single infinitesimally small source more than one image can be observed, each with magnification $\mu_j$. For microlensing events the angular sizes of the
images cannot be resolved. The only observable is the flux to which all nearby images contribute. A total magnification
\begin{equation}
 \mu_p = \sum_{j} \mu_j
\end{equation}
 can be defined and account for an increase/decrease of brightness from all images.


\subsection{Magnification of a point source near a fold}

The determinant of the Jacobian $A$ can have both positive and negative values depending in which region it is computed.Such regions are well defined and separated
by critical curves where the determinant vanishes. The lens equation maps the critical curves of the lens plane into caustics in the source plane. An example of the caustic 
shape can be seen in the magnification map of |||Figure 1|||.

||||||||||||||||||||
||||||Figure 1||||||
||||||||||||||||||||

Along these critical curves in the lens plane or caustics in the source plane,  the magnification $\mu$ is infinite, a result that simply follows from equation ( 
(3). This statement holds only fo infinitesimally small sources. For extended sources the maximum amplification is finite since only an infinitesimally small area of the source
overlaps with the caustics (Schneider Weiss 1986 paper others). 

The matrix $A$ at the coordinates of the caustic can have rank 1 or rank 0. If the rank is 1 then the coordinates belong to $fold$ or $cusp$ singularities. For the purpose 
of this paper we are interested only in the behaviour around $fold$ singularities. A condition necessary to distinguish between a fold and cusp singularities is that the
eigenvector of $A$ corresponding to the null eigenvalue is not a tangent vector of the critical curve (Dominik 2008, Schneider Weiss 1992).

||||||||||||||||||||
||||||Figure 2||||||
||||||||||||||||||||

The behaviour of the images of a point source and their corresponding magnifications near such a caustic has been thoroughly studied in the past.  


